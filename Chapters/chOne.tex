\chapter{Arithmetic}
\paragraph{} That night, Duckie decided to leave the village. He knew he would have to do so secretly, because the authorities did not approve of anyone leaving the village, much less searching for the Golden Goose. After all, they had tried so hard to make people forget about the Golden Goose altogether. Duckie waited until everyone had gone to sleep, and started on his journey.
\paragraph{} He began to sneak towards the village border, where he saw two guards snoring loudly. He tiptoed past, and all seemed quiet, but just as he turned his back to the village, he heard alarms ringing. 
\paragraph{} \textbf{Honk! Honk! Honk!}
\paragraph{} They droned. Duckie turned around and saw the two guards he had walked past coming towards him. The goose authorities had been alerted of his mission, and he would have to run. He began flying, and as he looked back, saw the goose police on his tail. 
\vfill
\pagebreak
%Addition
\subchapter{Addition}
{On the first night of his journey, Duckie flew \textbf{3} km from the village towards the mountain to try to escape the goose police. Unsure of how far the goose police would chase him, he decided to take a quick break to gain his strength, but in a flash, he saw the police on the horizon and began to fly again. From there, he flew another \textbf{6} km. Duckie wants to know how much he has traveled to know how far he is from the goose authorities. Where is Duckie?}
{He first flew \textbf{3} km, then changed this amount by \textbf{6}. This is \opadd{3}{6} km.}
{Addition is the most basic operation. It represents a change by a specific amount. Imagine an arrow in one direction, then putting another arrow on the end of the arrow. Addition lets us know where the end of the second arrow now is.}
{\begin{tikzpicture}
    \coordinate (A) at (0,0);
    \coordinate (B) at (0,3);
    \coordinate (C) at (0,9);
   
   \node [inner sep=0pt] at (C) {\includegraphics[height=0.8cm]{DuckieGami}};
    \draw[->,ultra thick,red, decorate, decoration={random steps,segment length=3pt,amplitude=0.2pt}] (A) -- (B) node[midway, left] {3 km};
    \draw[->,ultra thick,blue, decorate, decoration={random steps,segment length=3pt,amplitude=0.2pt}] (B) -- (C) node[midway, left] {6 km};
    
    \draw [decorate, 
	decoration = {calligraphic brace,mirror,
		raise=10pt,amplitude=5pt}] (A) -- (C) node[midway, right, xshift=0.5cm,line width=3pt] {9 km};
\end{tikzpicture}}
%Multiplication
\subchapter{Multiplication}
{Duckie did this routine on the first day, second, day, and third day.  Duckie wants to know how far he is from the goose village. Where is Duckie?}
{Duckie flew \textbf{9} km, \textbf{3} times. This is \opmul{3}{9} km.}
{Multiplication is the second basic operation. It represents stretching or squishing another number by a certain amount. Imagine a rubber band. Multiplying its length by P is the same as stretching the band so it is P long for every 1 length of the band.}
{\begin{tikzpicture}
    \coordinate (A) at (0,0);
    \coordinate (B) at (0,3);
    \coordinate (C) at (0,6);
    \coordinate (D) at (0,9);
    
    \node [inner sep=0pt] at (D) {\includegraphics[height=0.8cm]{DuckieGami}};
    \draw[->,ultra thick,red, decorate, decoration={random steps,segment length=3pt,amplitude=0.2pt}] (A) -- (B) node[midway, left] {9 km};
    \draw[->,ultra thick,blue, decorate, decoration={random steps,segment length=3pt,amplitude=0.2pt}] (B) -- (C) node[midway, left] {9 km};
    \draw[->,ultra thick,violet, decorate, decoration={random steps,segment length=3pt,amplitude=0.2pt}] (C) -- (D) node[midway, left] {9 km};
    
    \draw [decorate, 
	decoration = {calligraphic brace,mirror,
		raise=10pt,amplitude=5pt}] (A) -- (D) node[midway, right, xshift=0.5cm,line width=3pt] {27 km};
\end{tikzpicture}}
%Negatives
\subchapter{Negative Numbers}
{On the morning of the 4th day, after Duckie had traveled \textbf{P1} kms, he decided he had to take evasive maneuvers to try to confuse the goose police. In order to do this, he decided he should do the one thing they would not expect: go towards the village. He traveled backwards \textbf{3} km. This is called \textbf{-3} km. \linebreak – means opposite direction. Duckie wants to know how far he is from the village. Where is Duckie?}
{He first flew \textbf{27} km, then goes \textbf{-3} km. This is \opsub{27}{3} km.}
{A number less than 0 is called "negative", and is in the opposite direction. If sum of two numbers is 0, then they are called "additive inverses", because when added together their changes cancel each other out.}
{\begin{tikzpicture}
    \coordinate (A) at (0,0);
    \coordinate (B) at (0,9);
    \coordinate (C) at (1,9);
    \coordinate (D) at (1,8);
    \coordinate (E) at (2,0);
    \coordinate (F) at (2,8);
    
    \node [inner sep=0pt] at (F) {\includegraphics[height=0.8cm]{DuckieGami}};
    \draw[->,ultra thick,red, decorate, decoration={random steps,segment length=3pt,amplitude=0.2pt}] (A) -- (B) node[midway, right] {27 km};
    \draw[->,ultra thick,blue, decorate, decoration={random steps,segment length=3pt,amplitude=0.2pt}] (C) -- (D) node[midway, right] {-3 km};
    \draw[->,ultra thick,violet, decorate, decoration={random steps,segment length=3pt,amplitude=0.2pt}] (E) -- (F) node[midway, right] {24 km};
\end{tikzpicture}}
%Multiples of a negative
\subchapter{Multiples of a negative}
{While Duckie was traveling in the negative direction, he took a break every \textbf{-1} km. He flew that distance  \textbf{3} times. How far does Duckie travel while taking evasive maneuvers?}
{Duckie traveled \textbf{-1} km \textbf{3} times. This is $\opmul{3}{-1}$ km}
{Multiplying by a negative number shows stretching or squishing another number in the opposite direction}
{\begin{tikzpicture}
    \coordinate (A) at (0,0);
    \coordinate (B) at (0,9);
    \coordinate (C) at (1,9);
    \coordinate (D) at (1,8.666);
    \coordinate (E) at (1,8.333);
    \coordinate (F) at (1,8);
    \coordinate (G) at (2,0);
    \coordinate (H) at (2,8);
    
    \node [inner sep=0pt] at (H) {\includegraphics[height=0.8cm]{DuckieGami}};
    \draw[->,ultra thick,red, decorate, decoration={random steps,segment length=3pt,amplitude=0.2pt}] (A) -- (B) node[midway, right] {27 km};
    \draw[->,ultra thick,blue, decorate, decoration={random steps,segment length=3pt,amplitude=0.2pt}] (C) -- (D) node[midway, right] {-1};
    \draw[->,ultra thick,blue, decorate, decoration={random steps,segment length=3pt,amplitude=0.2pt}] (D) -- (E) node[midway, right] {-1};
    \draw[->,ultra thick,blue, decorate, decoration={random steps,segment length=3pt,amplitude=0.2pt}] (E) -- (F) node[midway, right] {-1};
     \draw[->,ultra thick,violet, decorate, decoration={random steps,segment length=3pt,amplitude=0.2pt}] (G) -- (H) node[midway, right] {24 km};
\end{tikzpicture}}
%Negative of a Negative
\subchapter{Negative of a Negative}
{On the fifth day, Duckie thought he had confused the goose police, and that they were no longer following him. To get to the Golden Goose, he knew he would have to travel away from the village, and so turned around again and flew for \textbf{6} km. Duckie traveled in the opposite direction of the negative direction by \textbf{6}, or \textbf{-(-6)}. Duckie traveled \textbf{6} km forward. Duckie wants to know how far he is from the goose village. Where is Duckie?}
{He was at \textbf{24} and then traveled \textbf{--6}. This is \opadd{24}{-(-6)} km.}
{A number in the opposite direction of the opposite direction of a number is in the same direction of that number. It is written as -(-number), which is the same as  that number.}
{\begin{tikzpicture}
    \coordinate (A) at (0,0);
    \coordinate (B) at (0,6);
    \coordinate (C) at (1,0);
    \coordinate (D) at (1,-6);
    \coordinate (E) at (2,0);
    \coordinate (F) at (2,6);
    
    \node [inner sep=0pt] at (F) {\includegraphics[height=0.8cm]{DuckieGami}};
    \draw[->,ultra thick,red, decorate, decoration={random steps,segment length=3pt,amplitude=0.2pt}] (A) -- (B) node[midway, right] {6 km};
    \draw[->,ultra thick,blue, decorate, decoration={random steps,segment length=3pt,amplitude=0.2pt}] (C) -- (D) node[midway, right] {-6 km};
    \draw[->,ultra thick,violet, decorate, decoration={random steps,segment length=3pt,amplitude=0.2pt}] (E) -- (F) node[midway, right] {- -6 km};
\end{tikzpicture}}
%Multiplication Table
\section{Multiplication Table}
\begin{table}[h]
\centering
\begin{tabular}{c| llllllllllll}
   & 0          & 1          & 2          & 3          & 4           & 5           & 6           & 7           & 8           & 9           & 10            \\
   \hline
0  & \textbf{0} & 0          & 0          & 0          & 0           & 0           & 0           & 0           & 0           & 0           & 0             \\
1  & 0          & \textbf{1} & 2          & 3          & 4           & 5           & 6           & 7           & 8           & 9           & 10            \\
2  & 0          & 2          & \textbf{4} & 6          & 8           & 10          & 12          & 14          & 16          & 18          & 20            \\
3  & 0          & 3          & 6          & \textbf{9} & 12          & 15          & 18          & 21          & 24          & 27          & 30            \\
4  & 0          & 4          & 8          & 12         & \textbf{16} & 20          & 24          & 28          & 32          & 36          & 40            \\
5  & 0          & 5          & 10         & 15         & 20          & \textbf{25} & 30          & 35          & 40          & 45          & 50            \\
6  & 0          & 6          & 12         & 18         & 24          & 30          & \textbf{36} & 42          & 48          & 54          & 60            \\
7  & 0          & 7          & 14         & 21         & 28          & 35          & 42          & \textbf{49} & 56          & 63          & 70            \\
8  & 0          & 8          & 16         & 24         & 32          & 40          & 48          & 56          & \textbf{64} & 72          & 80            \\
9  & 0          & 9          & 18         & 27         & 36          & 45          & 54          & 63          & 72          & \textbf{81} & 90            \\
10 & 0          & 10         & 20         & 30         & 40          & 50          & 60          & 70          & 80          & 90          & \textbf{100} 
\end{tabular}
\end{table}
\paragraph{Observations about the Multiplication Table}:
\linebreak
\paragraph{0} Any number multiplied by 0 is 0. This makes sense, because any number
repeated 0 times is the same as not having it at all.
\paragraph{1} Any number multiplied by 1 is itself. This makes sense, because any
number repeated one time is itself. This has a special name: \textit{Identity}
\paragraph{10} Any number multiplied by 10 is shifted to the left by one digit.
\paragraph{Symmetry} The table is identical across the bolded diagonal. This shows
that 4×2 is the same as 2×4. This property also has a special name:
\textit{Commutativity}.