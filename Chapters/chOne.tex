\chapter{Arithmetic}
\paragraph{} That night, Duckie couldn't sleep. He knew something had to be done. Looking out his window, he saw the border of the village and heard the goosepolice patrolling. He would try to find the golden goose, and would help create the great changes the hero was meant to bring. Hopefully then, he could save the village. He knew he would have to break the goose authority's rules and search for the hero. Duckie carefully crept out of his house. The authorities had made it difficult to leave the village. Not only was trying to leave heavily punished, it was also difficult. Duckie would have to be stealthy. 
\paragraph{} Duckie waited until everyone had gone to sleep, and started on his journey.
\paragraph{} He began to sneak towards the village border. As he approached, he saw two guards snoring loudly. He heard the fluttering of the wind, and could feel his feathers being lifted by the cold winds of early winter. He tiptoed past, and all seemed quiet for a moment.
\paragraph{} "Phew", he whispered as he crossed the border. 
\paragraph{} But just as he turned his back to the village, he heard alarms ringing. 
\paragraph{} \textbf{Honk! Honk! Honk!}
\paragraph{} They droned loudly. Duckie turned around and saw the two guards he had walked past coming towards him. The goose authorities had been alerted of his mission, and he would have to run. 
\paragraph{} "Hold it right there!"
\paragraph{} With a rustle of feathers and beating of wings, Duckie took off and began to fly towards the mountain as fast as he could. As he looked back, he saw the goose police on his tail. 
\vfill
\pagebreak
%Addition
\subchapter{Addition}
{\paragraph{} The first night of the journey, Duckie felt the breeze blowing beneath his wings as he flew away from the village. A thick fog had blanketed the forest surrounding the village, making it hard to fly. "Keep flying. Find the golden goose", Duckie said to himself, looking at the lights coming from the village houses. 
\paragraph{} To get away from the village, Duckie first flew $\distanceOneOpO km$. He turned around, hoping that he had shaken the goose authorities off the trail. Unable to see clearly through he fog, he decided to land on the forest brush to regain his strength, but just as he had settled down, he heard the nearing squawks of the goosepolice. With a spurt of energy, Duckie took off from the forest floor, and flew another $\distanceTwoOpO km$. As the sun rose, Duckie decided to form a plan. 
\paragraph{} He needed to know how much he had traveled in the direction of the mountain. Where was Duckie?}
{Duckie first flew $\distanceOneOpO km$, then changed this amount by $\distanceTwoOpO$. Duckie's position, $\distanceOneOpO$ kilometers, changed by $\distanceTwoOpO$ kilometers. This is $\distanceOneOpO + \distanceTwoOpO = \distanceThreeOpO km$.}
{Addition, or adding, is the most basic way of using numbers. It represents changing one number by another to form a single, combined number. It represents taking two separate quantities and combining them together to find a new quantity} 
{\begin{tikzpicture}
    \coordinate (A) at (0,0);
    \coordinate (B) at (0,3);
    \coordinate (C) at (0,9);

   \node [inner sep=0pt] at (C) {\includegraphics[height=0.8cm]{DuckieGami}};
    \draw[->,ultra thick,red, decorate, decoration={random steps,segment length=3pt,amplitude=0.2pt}] (A) -- (B) node[midway, left] {3 km};
    \draw[->,ultra thick,blue, decorate, decoration={random steps,segment length=3pt,amplitude=0.2pt}] (B) -- (C) node[midway, left] {6 km};

    \draw [decorate,
	decoration = {calligraphic brace,mirror,
		raise=10pt,amplitude=5pt}] (A) -- (C) node[midway, right, xshift=0.5cm,line width=3pt] {9 km};
\end{tikzpicture}
}
%Multiplication
\subchapter{Multiplication}
{\paragraph{} "Wow, they fly so quickly. I'm going to have to keep going." Duckie said aloud to himself. He heard the faint shouts of the goose police in the distance. 
\paragraph{} "Keep an eye out. You know what the goose authorities will do if we don't find him." they clamored.
\paragraph{} Duckie continued his journey toward the mountain, with the goose authorities following closely behind. \paragraph{} For the first 3 days, the same cycle would repeat. Duckie would fly $\distanceThreeOpO$ km. Each day, he stopped, hoping to find some rest. His muscles ached from exhaustion, but the thought of the golden goose forced him to keep going.
\paragraph{} Duckie wanted to know how much he had flown towards the mountain. Where was Duckie?}
{Duckie flew $\distanceThreeOpO~km$, $\distanceFourOpTw$ times. This means the amount he flew was triple what he flew on the first day. This is $\distanceThreeOpO \ast \distanceFourOpTw$, or $\distanceFiveOpTw km$.}
{Multiplication, or multiplying, is the second basic way of using numbers. It represents changing a number by a certain scale. For example, doubling, tripling, etc. $A\ast B$ is $A$ times more than $B$.}
{ \begin{tikzpicture}
    \coordinate (A) at (0,0);
    \coordinate (B) at (0,3);
    \coordinate (C) at (0,6);
    \coordinate (D) at (0,9);

    \node [inner sep=0pt] at (D) {\includegraphics[height=0.8cm]{DuckieGami}};
    \draw[->,ultra thick,red, decorate, decoration={random steps,segment length=3pt,amplitude=0.2pt}] (A) -- (B) node[midway, left] {$7 km$};
    \draw[->,ultra thick,blue, decorate, decoration={random steps,segment length=3pt,amplitude=0.2pt}] (B) -- (C) node[midway, left] {$7 km$};
    \draw[->,ultra thick,violet, decorate, decoration={random steps,segment length=3pt,amplitude=0.2pt}] (C) -- (D) node[midway, left] {$7 km$};

    \draw [decorate,
	decoration = {calligraphic brace,mirror,
		raise=10pt,amplitude=5pt}] (A) -- (D) node[midway, right, xshift=0.5cm,line width=3pt] {$21 km$};
\end{tikzpicture}
}
%Negatives
\subchapter{Negative Numbers}
{\paragraph{} On the morning of the fourth day, Duckie had traveled $\distanceFiveOpTw km$. His journey had gone successfully so far, but he knew he could not avoid the goosepolice much longer. He decided he had to take evasive maneuvers to try to confuse the goose police. 
\paragraph{} Their training to be the fastest and best in the village meant they would inevitably catch up with him if he kept flying without rest. To escape, he decided do the last thing they expected: go back towards the village. By the time Duckie had finished taking evasive maneuvers, he had traveled in the opposite direction of the mountain for $\distanceSixAbsOpTh km$. 
\paragraph{} Duckie wanted to know how far he was from the mountain. Where was Duckie?}
{This is called $\distanceSixOpTh km$. \linebreak – means opposite direction.  He first flew $\distanceFiveOpTw km$, then flew $\distanceSixOpTh km$. This is$ \distanceFiveOpTw - \distanceSixAbsOpTh$, or $\distanceSevenOpTh km$.}
{A number less than 0 is called "negative", and is in the opposite direction.}
{\begin{tikzpicture}
    \coordinate (A) at (0,0);
    \coordinate (B) at (0,9);
    \coordinate (C) at (1,9);
    \coordinate (D) at (1,8);
    \coordinate (E) at (2,0);
    \coordinate (F) at (2,8);

    \node [inner sep=0pt] at (F) {\includegraphics[height=0.8cm]{DuckieGami}};
    \draw[->,ultra thick,red, decorate, decoration={random steps,segment length=3pt,amplitude=0.2pt}] (A) -- (B) node[midway, right] {21 km};
    \draw[->,ultra thick,blue, decorate, decoration={random steps,segment length=3pt,amplitude=0.2pt}] (C) -- (D) node[midway, right] {-3 km};
    \draw[->,ultra thick,violet, decorate, decoration={random steps,segment length=3pt,amplitude=0.2pt}] (E) -- (F) node[midway, right] {18 km};
\end{tikzpicture}
}
%Multiples of a negative
\subchapter{Multiples of a negative}
{\paragraph{} As Duckie performed evasive maneuvers, he decided to be strategic. He knew he could not travel quickly, especially after having traveled so much. He needed to gather his strength for the difficult journey ahead, so he decided to take a break after every kilometer. He traveled -1 kilometer 3 times. 
\paragraph{} How far did Duckie travel while taking evasive maneuvers?}
{Duckie traveled $\mathbf{-1}~km$ $\distanceSixAbsOpTh$ times. This is ${-1}\ast\distanceSixAbsOpTh$, or $\distanceSixOpTh$ km.}
{Multiplying by a negative number shows stretching or scaling a number by some amount, but in the opposite direction.}
{ \begin{tikzpicture}
    \coordinate (A) at (0,0);
    \coordinate (B) at (0,9);
    \coordinate (C) at (1,9);
    \coordinate (D) at (1,8.666);
    \coordinate (E) at (1,8.333);
    \coordinate (F) at (1,8);
    \coordinate (G) at (2,0);
    \coordinate (H) at (2,8);

    \node [inner sep=0pt] at (H) {\includegraphics[height=0.8cm]{DuckieGami}};
    \draw[->,ultra thick,red, decorate, decoration={random steps,segment length=3pt,amplitude=0.2pt}] (A) -- (B) node[midway, right] {21 km};
    \draw[->,ultra thick,blue, decorate, decoration={random steps,segment length=3pt,amplitude=0.2pt}] (C) -- (D) node[midway, right] {-1};
    \draw[->,ultra thick,blue, decorate, decoration={random steps,segment length=3pt,amplitude=0.2pt}] (D) -- (E) node[midway, right] {-1};
    \draw[->,ultra thick,blue, decorate, decoration={random steps,segment length=3pt,amplitude=0.2pt}] (E) -- (F) node[midway, right] {-1};
     \draw[->,ultra thick,violet, decorate, decoration={random steps,segment length=3pt,amplitude=0.2pt}] (G) -- (H) node[midway, right] {18 km};
\end{tikzpicture}
}
%Negative of a Negative
\subchapter{Negative of a Negative}
{
	\paragraph{} By the fifth day, Duckie could no longer heard the police in the distance. In a way, he felt bad for them - they had been forced to fight him, whether or not the believed in the prophecy of the golden goose. Feeling he had avoided the goose police, Duckie began to head back towards the mountain. 
	\paragraph{} "I've got to make up for lost time. The goose police are really tough." 
	\paragraph{} Duckie turned away from the village, and began to fly as long and far as he could. After $\distanceEightAbsOpFo~km$, Duckie felt faint. Flying so far was not an easy task. He had flown in the opposite direction of the negative direction for $\distanceEightAbsOpFo~km$, or $\mathbf{-(\distanceEightOpFo)}$. 
	\paragraph{} Duckie wanted to know how far he was from the mountain. Where was Duckie?
}
{Initially, Duckie was at $\distanceSevenOpTh~km$. He then traveled $-(\distanceEightOpFo)~km$. Duckie traveled $\distanceEightAbsOpFo~km$ towards the mountain. he had flown $\distanceSevenOpTh+\distanceEightAbsOpFo~km$, or $\distanceNineOpFi~km$ towards the mountain.}
{The opposite direction of the opposite or negative direction of a number is in the same direction of the number. It is written as -(-number), which is the same as the number itself.}
{ \begin{tikzpicture}
    \coordinate (A) at (0,0);
    \coordinate (B) at (0,6);
    \coordinate (C) at (1,0);
    \coordinate (D) at (1,-6);
    \coordinate (E) at (2,0);
    \coordinate (F) at (2,6);

    \node [inner sep=0pt] at (F) {\includegraphics[height=0.8cm]{DuckieGami}};
    \draw[->,ultra thick,red, decorate, decoration={random steps,segment length=3pt,amplitude=0.2pt}] (A) -- (B) node[midway, right] {$6 km$};
    \draw[->,ultra thick,blue, decorate, decoration={random steps,segment length=3pt,amplitude=0.2pt}] (C) -- (D) node[midway, right] {$-6 km$};
    \draw[->,ultra thick,violet, decorate, decoration={random steps,segment length=3pt,amplitude=0.2pt}] (E) -- (F) node[midway, right] {$- -6 km$};
\end{tikzpicture}
}
%Multiplication Table
\section{Multiplication Table}
\begin{table}[h]
\centering
\begin{tabular}{c| llllllllllll}
   & 0          & 1          & 2          & 3          & 4           & 5           & 6           & 7           & 8           & 9           & 10            \\
   \hline
0  & \textbf{0} & 0          & 0          & 0          & 0           & 0           & 0           & 0           & 0           & 0           & 0             \\
1  & 0          & \textbf{1} & 2          & 3          & 4           & 5           & 6           & 7           & 8           & 9           & 10            \\
2  & 0          & 2          & \textbf{4} & 6          & 8           & 10          & 12          & 14          & 16          & 18          & 20            \\
3  & 0          & 3          & 6          & \textbf{9} & 12          & 15          & 18          & 21          & 24          & 27          & 30            \\
4  & 0          & 4          & 8          & 12         & \textbf{16} & 20          & 24          & 28          & 32          & 36          & 40            \\
5  & 0          & 5          & 10         & 15         & 20          & \textbf{25} & 30          & 35          & 40          & 45          & 50            \\
6  & 0          & 6          & 12         & 18         & 24          & 30          & \textbf{36} & 42          & 48          & 54          & 60            \\
7  & 0          & 7          & 14         & 21         & 28          & 35          & 42          & \textbf{49} & 56          & 63          & 70            \\
8  & 0          & 8          & 16         & 24         & 32          & 40          & 48          & 56          & \textbf{64} & 72          & 80            \\
9  & 0          & 9          & 18         & 27         & 36          & 45          & 54          & 63          & 72          & \textbf{81} & 90            \\
10 & 0          & 10         & 20         & 30         & 40          & 50          & 60          & 70          & 80          & 90          & \textbf{100} 
\end{tabular}
\end{table}
\paragraph{Observations about the Multiplication Table}:
\linebreak
\paragraph{0} Any number multiplied by 0 is 0. This makes sense, because any number
repeated 0 times is the same as not having it at all.
\paragraph{1} Any number multiplied by 1 is itself. This makes sense, because any
number repeated one time is itself. This has a special name: \textit{Identity}
\paragraph{10} Any number multiplied by 10 is shifted to the left by one digit.
\paragraph{Symmetry} The table is identical across the diagonal. This shows
that 4$\ast$2 is the same as 2$\ast$4. This property also has a special name:
\textit{Commutativity}.