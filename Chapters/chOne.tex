\chapter{Arithmetic}
\paragraph{} That night, Duckie decided to leave the village. He knew he would leave secretly because of the tyrannical rules of the goose authorieties. They had tried so hard to keep their power, and the search for the Golden Goose risked everything they stood for. Duckie waited until everyone had gone to sleep, and started on his journey.
\paragraph{} He began to sneak towards the village border, where he saw two guards snoring loudly. He tiptoed past, and all seemed quiet, but just as he turned his back to the village, he heard alarms ringing. 
\paragraph{} \textbf{Honk! Honk! Honk!}
\paragraph{} They droned. Duckie turned around and saw the two guards he had walked past coming towards him. The goose authorities had been alerted of his mission, and he would have to run. He began flying, and as he looked back, saw the goose police on his tail. 
\vfill
\pagebreak
%Addition
\subchapter{Addition}
{On the first night of his journey, Duckie flew $\mathbf{3} km$ from the village towards the mountain with the goose police in pursuit. Looking back, he wouldn't see them behind him, so, unsure of how far the goose police would chase him, he decided to take a quick break to gain his strength, but in a flash, he saw the police on the horizon and began to fly again. From there, he flew another $\mathbf{4} km$. Duckie wanted to know how much he had traveled in the direction of the mountain. Where was Duckie?}
{He first flew $\mathbf{3} km$, then changed this amount by $\mathbf{4}$. Duckie's position, 3 kilometers, changed by 4 kilometers. This is $\opadd{3}{4} km$.}
{Addition, or adding, is the most basic way of using numbers. It represents changing one number by another to form a single, combined number.} 
{\begin{tikzpicture}
    \coordinate (A) at (0,0);
    \coordinate (B) at (0,3);
    \coordinate (C) at (0,9);

   \node [inner sep=0pt] at (C) {\includegraphics[height=0.8cm]{DuckieGami}};
    \draw[->,ultra thick,red, decorate, decoration={random steps,segment length=3pt,amplitude=0.2pt}] (A) -- (B) node[midway, left] {3 km};
    \draw[->,ultra thick,blue, decorate, decoration={random steps,segment length=3pt,amplitude=0.2pt}] (B) -- (C) node[midway, left] {6 km};

    \draw [decorate,
	decoration = {calligraphic brace,mirror,
		raise=10pt,amplitude=5pt}] (A) -- (C) node[midway, right, xshift=0.5cm,line width=3pt] {9 km};
\end{tikzpicture}
}
%Multiplication
\subchapter{Multiplication}
{As Duckie began his journey towards the mountain, the goose authorities followed close behind. Every time he turned back, he saw them on the horizon. On each of the first three days, he flew $\mathbf{7km}$. Duckie wanted to know how much he had flown towards the mountain. Where was Duckie?}
{Duckie flew $\mathbf{7} km$, $\mathbf{3} times$. This means the amount he flew was triple what he flew on the first day. This is $\opmul{3}{7} km$.}
{Multiplication, or multiplying, is the second basic way of using numbers. It represents changing a number by a certain scale. For example, doubling, tripling, etc. $A\ast B$ is $A$ times more than $B$.}
{ \begin{tikzpicture}
    \coordinate (A) at (0,0);
    \coordinate (B) at (0,3);
    \coordinate (C) at (0,6);
    \coordinate (D) at (0,9);

    \node [inner sep=0pt] at (D) {\includegraphics[height=0.8cm]{DuckieGami}};
    \draw[->,ultra thick,red, decorate, decoration={random steps,segment length=3pt,amplitude=0.2pt}] (A) -- (B) node[midway, left] {$7 km$};
    \draw[->,ultra thick,blue, decorate, decoration={random steps,segment length=3pt,amplitude=0.2pt}] (B) -- (C) node[midway, left] {$7 km$};
    \draw[->,ultra thick,violet, decorate, decoration={random steps,segment length=3pt,amplitude=0.2pt}] (C) -- (D) node[midway, left] {$7 km$};

    \draw [decorate,
	decoration = {calligraphic brace,mirror,
		raise=10pt,amplitude=5pt}] (A) -- (D) node[midway, right, xshift=0.5cm,line width=3pt] {$21 km$};
\end{tikzpicture}
}
%Negatives
\subchapter{Negative Numbers}
{On the morning of the fourth day, Duckie had traveled $\mathbf{21} km$. He decided he had to take evasive maneuvers to try to confuse the goose police. Their training to be the fastest and best in the village meant they would inevitably catch up with him. To escape, he decided do the last thing they expected: go back towards the village. He traveled in the opposite direction of the mountain for $\mathbf{3} km$. Duckie wanted to know how far he was from the mountain. Where was Duckie?}
{This is called $\mathbf{-3} km$. \linebreak – means opposite direction.  He first flew $\mathbf{21} km$, then flew $\mathbf{-3} km$. This is$ \opsub{21}{3} km$.}
{A number less than 0 is called "negative", and is in the opposite direction.}
{\begin{tikzpicture}
    \coordinate (A) at (0,0);
    \coordinate (B) at (0,9);
    \coordinate (C) at (1,9);
    \coordinate (D) at (1,8);
    \coordinate (E) at (2,0);
    \coordinate (F) at (2,8);

    \node [inner sep=0pt] at (F) {\includegraphics[height=0.8cm]{DuckieGami}};
    \draw[->,ultra thick,red, decorate, decoration={random steps,segment length=3pt,amplitude=0.2pt}] (A) -- (B) node[midway, right] {21 km};
    \draw[->,ultra thick,blue, decorate, decoration={random steps,segment length=3pt,amplitude=0.2pt}] (C) -- (D) node[midway, right] {-3 km};
    \draw[->,ultra thick,violet, decorate, decoration={random steps,segment length=3pt,amplitude=0.2pt}] (E) -- (F) node[midway, right] {18 km};
\end{tikzpicture}
}
%Multiples of a negative
\subchapter{Multiples of a negative}
{Duckie was tried from traveling so much in the past few days. As he traveled in the negative direction, to try to gain his strength for the difficult journey ahead, he took a break every $\mathbf{-1} km$. He flew that distance $\mathbf{3}$ times. How far did Duckie travel while taking evasive maneuvers?}
{Duckie traveled $\mathbf{-1} km$ $\mathbf{3}$ times. This is -$\opmul{3}{1}$ km}
{Multiplying by a negative number shows stretching or scaling a number by some amount, but in the opposite direction.}
{ \begin{tikzpicture}
    \coordinate (A) at (0,0);
    \coordinate (B) at (0,9);
    \coordinate (C) at (1,9);
    \coordinate (D) at (1,8.666);
    \coordinate (E) at (1,8.333);
    \coordinate (F) at (1,8);
    \coordinate (G) at (2,0);
    \coordinate (H) at (2,8);

    \node [inner sep=0pt] at (H) {\includegraphics[height=0.8cm]{DuckieGami}};
    \draw[->,ultra thick,red, decorate, decoration={random steps,segment length=3pt,amplitude=0.2pt}] (A) -- (B) node[midway, right] {21 km};
    \draw[->,ultra thick,blue, decorate, decoration={random steps,segment length=3pt,amplitude=0.2pt}] (C) -- (D) node[midway, right] {-1};
    \draw[->,ultra thick,blue, decorate, decoration={random steps,segment length=3pt,amplitude=0.2pt}] (D) -- (E) node[midway, right] {-1};
    \draw[->,ultra thick,blue, decorate, decoration={random steps,segment length=3pt,amplitude=0.2pt}] (E) -- (F) node[midway, right] {-1};
     \draw[->,ultra thick,violet, decorate, decoration={random steps,segment length=3pt,amplitude=0.2pt}] (G) -- (H) node[midway, right] {18 km};
\end{tikzpicture}
}
%Negative of a Negative
\subchapter{Negative of a Negative}
{On the fifth day, feeling that he had avoided the goose police and confused them, continued to head towards the mountain. To get to the Golden Goose, he knew he would have to travel away from the village, and so turned around and flew for $\mathbf{6} km$. Duckie traveled in the opposite direction of the negative direction by $\mathbf{6}$, or $\mathbf{-(-6)}$. Duckie wanted to know how far he was from the mountain. Where was Duckie?}
{Initially, Duckie was at $\mathbf{18 km}$. He then traveled $\mathbf{-(-6)}$. Duckie traveled $\mathbf{6} km$ towards the mountain. he had flown $\opadd{18}{6} km$ towards the mountain.}
{The opposite direction of the opposite or negative direction of a number is in the same direction of the number. It is written as -(-number), which is the same as the number itself.}
{ \begin{tikzpicture}
    \coordinate (A) at (0,0);
    \coordinate (B) at (0,6);
    \coordinate (C) at (1,0);
    \coordinate (D) at (1,-6);
    \coordinate (E) at (2,0);
    \coordinate (F) at (2,6);

    \node [inner sep=0pt] at (F) {\includegraphics[height=0.8cm]{DuckieGami}};
    \draw[->,ultra thick,red, decorate, decoration={random steps,segment length=3pt,amplitude=0.2pt}] (A) -- (B) node[midway, right] {$6 km$};
    \draw[->,ultra thick,blue, decorate, decoration={random steps,segment length=3pt,amplitude=0.2pt}] (C) -- (D) node[midway, right] {$-6 km$};
    \draw[->,ultra thick,violet, decorate, decoration={random steps,segment length=3pt,amplitude=0.2pt}] (E) -- (F) node[midway, right] {$- -6 km$};
\end{tikzpicture}
}
%Multiplication Table
\section{Multiplication Table}
\begin{table}[h]
\centering
\begin{tabular}{c| llllllllllll}
   & 0          & 1          & 2          & 3          & 4           & 5           & 6           & 7           & 8           & 9           & 10            \\
   \hline
0  & \textbf{0} & 0          & 0          & 0          & 0           & 0           & 0           & 0           & 0           & 0           & 0             \\
1  & 0          & \textbf{1} & 2          & 3          & 4           & 5           & 6           & 7           & 8           & 9           & 10            \\
2  & 0          & 2          & \textbf{4} & 6          & 8           & 10          & 12          & 14          & 16          & 18          & 20            \\
3  & 0          & 3          & 6          & \textbf{9} & 12          & 15          & 18          & 21          & 24          & 27          & 30            \\
4  & 0          & 4          & 8          & 12         & \textbf{16} & 20          & 24          & 28          & 32          & 36          & 40            \\
5  & 0          & 5          & 10         & 15         & 20          & \textbf{25} & 30          & 35          & 40          & 45          & 50            \\
6  & 0          & 6          & 12         & 18         & 24          & 30          & \textbf{36} & 42          & 48          & 54          & 60            \\
7  & 0          & 7          & 14         & 21         & 28          & 35          & 42          & \textbf{49} & 56          & 63          & 70            \\
8  & 0          & 8          & 16         & 24         & 32          & 40          & 48          & 56          & \textbf{64} & 72          & 80            \\
9  & 0          & 9          & 18         & 27         & 36          & 45          & 54          & 63          & 72          & \textbf{81} & 90            \\
10 & 0          & 10         & 20         & 30         & 40          & 50          & 60          & 70          & 80          & 90          & \textbf{100} 
\end{tabular}
\end{table}
\paragraph{Observations about the Multiplication Table}:
\linebreak
\paragraph{0} Any number multiplied by 0 is 0. This makes sense, because any number
repeated 0 times is the same as not having it at all.
\paragraph{1} Any number multiplied by 1 is itself. This makes sense, because any
number repeated one time is itself. This has a special name: \textit{Identity}
\paragraph{10} Any number multiplied by 10 is shifted to the left by one digit.
\paragraph{Symmetry} The table is identical across the bolded diagonal. This shows
that 4×2 is the same as 2×4. This property also has a special name:
\textit{Commutativity}.