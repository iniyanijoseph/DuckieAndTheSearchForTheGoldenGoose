\chapter{Polynomials}
\paragraph{} The next morning, Duckie and G{\"u}s looked out of their campground. 
\paragraph{} "Looks like we are almost there. I've been talking to Maggie and I think she wants to come with us to find the Golden Goose" G{\"u}s remarked.
\paragraph{} "That's great!" Duckie said. "We shouldn't count our geese before they hatch though. Even though we are almost there, we are going to start the most difficult part of the journey. Look at the mountain!"
\paragraph{} The looked at the mountain and saw it towering above them. 
\paragraph{} "I agree. I think we need to prepare."
\paragraph{} As the rest of the geese woke up, Duckie and G{\"u}s stood up and announced their departure. 
\paragraph{} "Hello everyone. Thank you for your hospitality. I want to make sure we can get there quickly so that we can save our village, so we need to travel fast. We need to continue on our journey. Would anyone like to come with us?"
\paragraph{} Maggie stepped forward. "I'll join!"
\paragraph{} "Great!" Duckie said. 
\vfill
\pagebreak
\subchapter{Polynomial Functions}
{\paragraph{} Duckie, Maggie, and G{\"u}s began flying towards the mountain. 
\paragraph{} "Wow thats pretty steep. We should probably try to figure out where it is and what it looks like." Maggie noted. 
\paragraph{} "That makes sense" Duckie said. "Looking at the mountain, it looks to me like it touches the ground at two points. But how will we figure out how the mountain looks?"
\paragraph{} Maggie pondered this over this for a moment.
\paragraph{} "What if we made a function? It can map every position on the ground to the height of the mountain at that point. Since it touches the ground at two points, h(x) = 0"
\paragraph{} "That's a good idea" Gus said. "But where does it touch the ground? Maybe we can say it touches the ground when $x_1$ = ? and $x_2$ = ?"
\paragraph{} What is h(x)?}
{
	\paragraph{} We can separate the function into two parts. We want the function to be 0 when x is 0. 
	$x-x_1$ will be 0 when $x = x_1$. Similarly, $x-x_2$ will be 0 when $x=x_2$. 
	\paragraph{} $0\ast$any number is 0. 
	\paragraph{} Because of this, we can set $h(x)=(x-x_1)(x-x_2)$.
}
{A polynomial function is a function where f(x) is the sum of terms, such that each term is a constant multiplied with x raised to a natural number. In other words, $\sum a*x^{k}$ s.t. $a \in \mathbb{R}, k \in \mathbb{N}$.}
{}

\subchapter{Polynomial Functions}
{\paragraph{} Duckie thought for a moment. 
\paragraph{} "I am not so sure about that actually. To me, the function looks like $ax^2-bx+c$"
\paragraph{} Maggie was hesitant "Won't this give us the incorrect ground locations?" 
\paragraph{} "I'm not sure. How can we make sure?"
\paragraph{} Where would the mountain touch the ground if $h(x)=ax^2-bx+c$?
}
{\paragraph{} $\frac{-b \pm \sqrt{b^2-4ac}}{2a} = $
\paragraph{} $\frac{-b \pm \sqrt{b^2-4ac}}{2a} = $
\paragraph{} $\frac{-b \pm \sqrt{b^2-4ac}}{2a} = $}
{The roots of a polynomial with degree two are $\frac{-b \pm \sqrt{b^2-4ac}}{2a}$}
{}

\subchapter{Polynomial Functions}
{ The trio sped up as they approached the mountain. As they approached, it seemed to grow taller and taller and wider and wider. 
	They could see the snowcapped peaks clearly, and realized they had made the mistake. 
	The mountain was x meters taller than they had originally thought. 
	\paragraph{} "What can we do?" Gus pondered aloud.
	\paragraph{} "Can we modify the function we made before?" Maggie asked.
	\paragraph{} Modify the height of the function so that every point is x points higher.
}
{
	$h(x) = h(x) + x$
}
{ A function can be transformed in 3 ways.
	\begin{enumerate}
		\item It can be shifted up and down by adding a constant from the overall function
		\item It can be shifted to the left and right by subtracting a constant from x
		\item It can be compressed by multiplying x by a constant.
	\end{enumerate}
}
{}

What is an exponent?\\
They have not seen the top of the mountain yet, so they assume that it is a quadratic function\\
What is a polynomial? They need to figure out what the mountain looks like so that they can reason about it. The function has a peak, and has height 0 when on the ground, its positions are ($x_1$, $x_2$)\\
\indent In short, they must find a polynomial given the roots \\
How do we do it the other way around? - Finding the roots of a polynomial - Find roots given quadratic formula\\
As they fly around the mountain, they see the second peak. They guess the new roots to be $x_3$ and $x_4$ (and also $x_1$, $x_2$)\\
Transformations - they realize they guessed wrong as they approach the mountain, so must adjust the function\\
We must expand the story of the golden goose to include advice (which Duckie uses through they story like in that dragon book). The legend will speak of a riddlye which will unlock the secrets of the mountain a)  I have to beef up the prologue b) The riddle will give some hits on exponentiation and will help provide information on the journey.
\pagebreak