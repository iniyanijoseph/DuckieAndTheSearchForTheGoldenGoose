\chapter{Ratios and Fractions}
\paragraph{} It took a lot of work, but the authorities finally figured out where Duckie was. They noticed that Duckie and Güs made quite a bit of progress, which was very bad news for them. They wanted to use this information to try to stop them. Because of Güs defecting to join Duckie's journey, they realized that sending other policegeese would make it hard to keep their power in the village. 
\paragraph{} Through their past migrations, they knew lots of bounty hunters along the way, and they believed that they could have them set traps. These bounty hunters were famously clever, and each of them decided to use specific strategies to help them catch Duckie and Güs.
\vfill
\pagebreak
%Ratios
\subchapter{Ratios}
{The first bounty hunter decided to look at the entire journey, and set up traps every $\textbf{40} km$. He estimated that the entire journey was $\textbf{120} km$ long. Duckie having guessed this fairly obvious scheme,  thought that if he knew how many traps there would be, he could avoid them. For every km in the entire journey, there would be a trap every $\textbf{40} km$. How many traps will there be on the journey?}
{For every $\textbf{40} km$ Duckie has to travel out of $\textbf{120} km$ overall, there will be a trap. That means that there is a trap every $\frac{120}{40} km$ of the journey. For every one km in $\textbf{40} km$, there are $\textbf{3} km$ in the journey, which means that there are $\textbf{3}$ lengths of size $\textbf{40} km$ within $\textbf{120} km$. There are $\frac{3}{1}$, or $\textbf{3}$ traps.}
{A ratio is a certain amount for each of another amount, and is written as $\frac{A}{B}$. Break up A into B equal parts.}
{\input{Chapters/chFourGraphics/Ratios.tikz}}
%Multiplying with Fractions
\subchapter{Multiplying with Fractions}
{Using this knowledge, Duckie figured out how to avoid the first bounty hunter's traps. The second bounty hunter decided to prepare better than the first hunter. He realized that km are a very large unit for geese, so decided to start measuring the distance Duckie and Güs flew in a unit he created called "goosemeters" (gm). This meant that he needed to change Duckie and Güs's position from kilometers into goosemeters. There are exactly $7 gm$ in every $3 km$. To help beat the second bounty hunter, Duckie and Güs decide to also measure using $gm$. How many gms are in $40 km$?}{
$\frac{7 gm}{3 km}\ast\frac{120 km}{1}\ast\frac{1}{3}=\frac{280}{3} gm$.}
{Like any number, multiplying by a ratio represents scaling by a certain amount. Imagine a rubber band which is P long, which you can stretch and squish. Multiplication represents scaling the band such that $P\ast\frac{A}{B}$ is the new length of the band. This is the same as $\frac{P\ast A}{B}$, or $\frac{P}{B}\ast A$. This can be seen as dividing the band into B equal parts, then taking one of those parts A times. \paragraph{} Notice that that $\frac{A}{B}$ can create numbers in between 0, 1, 2, etc.}{MultiplyingFractions.png}
%Comparing Fractions
\subchapter{Comparing Fractions}
{Duckie thought that if he traveled enough in one day, the bounty hunter would become confused and leave them alone. On the eighth day, Duckie and Güs traveled an additional $\frac{1}{5}$ of the journey. If Duckie traveled more on the eighth day than the combined previous seven days ($\frac{1}{3}$ of the journey), the bounty hunter would become confused and stop chasing Duckie. Has Duckie managed to confuse the bounty hunter?}{Duckie realizes that \begin{center}
    $\frac{5}{5}=1 \linebreak
    \frac{1}{3}\ast 1 = \frac{1}{3} \linebreak
    \frac{1}{3}\ast\frac{5}{5} = \frac{1\ast 5}{3\ast 5} = \frac{5}{15}\linebreak  $Which means $\frac{1}{3}$  is same as  $\frac{5}{15} \linebreak\linebreak
    \textit{similarly}
    \frac{1}{5} = \frac{1}{5}\ast\frac{3}{3} = \frac{3}{15}$
\end{center}
\paragraph{} Now, we can compare the equal size parts together. $\frac{3}{15} < \frac{5}{15}$. Duckie traveled less on the eighth day than the other days combined.}{We can only deal with fractions together using same-size parts. We can chop up one fraction by multiplying it by another fraction equal to one. If the \textit{denominator}s (the bottom number of the fractions) are the same, the fractions have same sized parts.}
{ComparingFractions}
%Adding Fractions
\subchapter{Adding Fractions}
{Despite Duckie's best efforts, he was unfortunately not able to confuse the bounty hunter. He was starting to get worried, but Güs had a genius idea. If they could find out where they were along the journey, they might have been able find a new route, which would take them away from the hunter's grasp! They traveled $\frac{1}{3}$, then they traveled $\frac{1}{5}$. How much of the journey have Duckie and Güs traveled?}{Duckie and Güs have traveled
$\frac{1}{3}+\frac{1}{5}\linebreak=\frac{3+5}{15}\linebreak=\frac{8}{15}$ of the journey.}{When adding fractions, break up each fraction so that they are in terms of same-size parts (or having a common denominator). Then, add the numerators as usual.}{AddingFractions}
\paragraph{} Duckie and Güs sighed in relief. After running for bounty hunters for two days and two nights, they were exhausted. They decided to travel slower for a few days and enjoy the scenery. 
\paragraph{} "I can't believe how persistent the goose authorities are." Duckie sighed. "I knew this journey would be tough, but I wasn't expecting bounty hunters!"
\paragraph{} Güs considered this. He had worked with the goose authorities, and knew their patterns. He realized how unusual this was. "I agree, the authorities must really be desparate".
\paragraph{} Just as Duckie opened his beak to reply, he felt a sudden weight on his shoulders, and began spiraling out of control. It was a net! Another bounty hunter must have set a trap! He started falling down fast, gaining speed as he went. \textbf{Crash}!. Duckie fell to the ground with a thud. Güs hurriedly landed next to him. 
\paragraph{} "Are you alright Duckie‽" he cried in alarm.
\paragraph{} "I'm all right, but I think I need to rest my wing for a few days", Duckie winced. He felt that he hadn't broken a wing, but saw some bruising, and knew he would need time to heal. 
 \paragraph{} "The forest is dangerous though! We can't stay here in the open with all the predators!"
 \paragraph{} "I think my friend Bessie the cow's farm is close enough to walk to from here," Duckie said. "I think she will be able to help keep us safe from the goose authorities."
 \paragraph{} Together, they waddled to Bessie's farm.