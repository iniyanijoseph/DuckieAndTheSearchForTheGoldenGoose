\chapter{Ratios and Fractions}
\paragraph{} It took a lot of work, but the authorities finally figured out where Duckie and G{\"u}s were. They noticed that the pair had made quite a bit of progress, and that they needed to take the threat seriously. Duckie and G{\"u}s's position information was know extremely valuable to stopping them. G{\"u}s, a prominent policegoose's defection, showed them that sending other policegeese would only undo their efforts in building a loyal army.  
\paragraph{} In their past migrations, they had worked with the worst geese in the land to help control the village. As a result, they knew many bounty hunters along the way, and believed that they could have them set traps. 
\paragraph{} They quickly called them all, and sent them to bring back Duckie and G{\"u}s, dead or alive. These famously clever bounty hunters each decided to strategy to help them catch Duckie and G{\"u}s.
\vfill
\pagebreak
%Ratios
\subchapter{Ratios}
{The first bounty hunter decided to look at the entire journey, and set up traps every $\distanceThreeTh~km$. He guessed that the entire journey was $\distanceFourFoO~km$ long. Duckie having guessed the first hunter's (fairly obvious) scheme, knew that if he knew the number of traps, he could avoid them. For every $\distanceThreeTh~km$ in the journey, there was $1$ trap. How many traps were there be on the journey?}
{For every $\distanceThreeTh~km$ Duckie had to travel out of $\distanceThreeTh~km$ overall, there was one trap. That meant that there was a trap in every $\frac{\distanceFourFoO}{\distanceThreeTh}~km$ of the journey. For every one km in $\distanceThreeTh~km$, there were $\distanceFourFoTw~km$ in the journey, which meant that there were $\textbf{3}$ lengths of size $\distanceThreeTh~km$ within $\distanceFourFoO~km$. There were $\frac{\distanceFourFoTw}{1}$, or $\distanceFourFoTw$ traps.}
{A ratio represents the quantity of A compared to the quantity of B, and is written as $\frac{A}{B}$, or $A$ for every $B$. This is also the same as breaking up A into B equal parts.}
{\input{Chapters/chFourGraphics/Ratios.tikz}}
%Multiplying with Fractions
\subchapter{Multiplying with Fractions}
{Using his knowledge of mathematics, Duckie figured out how to avoid the first bounty hunter's traps. Never underestimating his opponent, the second bounty hunter decided to prepare better. He realized that $km$ are a very large unit for geese, so decided to start measuring the distance Duckie and G{\"u}s flew in a unit he created called "goosemeters" ($gm$). This meant that he needed to change Duckie and G{\"u}s's position from kilometers into goosemeters. There are exactly $7 gm$ in every $3 km$. To help beat the second bounty hunter, Duckie and G{\"u}s decided to also measure using $gm$. How many $gm$s are in $40 km$?}
{$\frac{7 gm}{3 km}\ast\frac{120 km}{1}\ast\frac{1}{3}=\frac{280}{3} gm$.}
{Multiplying by a ratio also represents stretching or scaling by a certain amount. It represents scaling the number $P$ so that $P\ast\frac{A}{B}$ is its new size. This is the same as breaking up $P\ast A$ into $B$ equal parts, or $\frac{P}{B}$ $A$ times (breaking up $P$ into $B$ parts and taking each part $A$ times).\paragraph{} Notice that that the size of $\frac{A}{B}$ can be in between 0, 1, 2, etc.}
{\begin{tikzpicture}
    \coordinate (A) at (0,0);
    \coordinate (Ao) at (0,0.6);
    \coordinate (At) at (0,1.2);
    \coordinate (Ath) at (0,1.8);
    \coordinate (Af) at (0,2.4);
    \coordinate (B) at (0,3);
    \coordinate (C) at (0,6);
    \coordinate (D) at (0,9);
    
    \node [inner sep=0pt] at (D) {\includegraphics[height=0.8cm]{DuckieGami}};
    \draw[-|,ultra thick,purple, decorate, decoration={random steps,segment length=3pt,amplitude=0.2pt}] (A) -- (Ao) node[midway, left] {$\frac{40}{3} km$};
    \draw[->,ultra thick,red, decorate, decoration={random steps,segment length=3pt,amplitude=0.2pt}] (Ao) -- (At) node[midway, left] {$\frac{40}{3} km$};
    \draw[->,ultra thick,red, decorate, decoration={random steps,segment length=3pt,amplitude=0.2pt}] (At) -- (Ath) node[midway, left] {$\frac{40}{3} km$};
    \draw[->,ultra thick,red, decorate, decoration={random steps,segment length=3pt,amplitude=0.2pt}] (Ath) -- (Af) node[midway, left] {$\frac{40}{3} km$};
    \draw[-|,ultra thick,red, decorate, decoration={random steps,segment length=3pt,amplitude=0.2pt}] (Af) -- (B) node[midway, left] {$\frac{40}{3} km$};
    \draw[->,ultra thick,blue, decorate, decoration={random steps,segment length=3pt,amplitude=0.2pt}] (B) -- (C) node[midway, left] {$40 km$};
    \draw[->,ultra thick,blue, decorate, decoration={random steps,segment length=3pt,amplitude=0.2pt}] (C) -- (D) node[midway, left] {$40 km$};
    
    \foreach \pos in {1,...,7} {	
        \draw[->,ultra thick,violet, decorate, decoration={random steps,segment length=3pt,amplitude=0.2pt}] (1.5,\pos*0.6 - 0.6) -- (1.5,\pos *0.6) node[midway, left] {$\frac{40}{3} km$};
    }

    \draw [decorate, 
	decoration = {calligraphic brace,mirror,
		raise=10pt,amplitude=5pt}] (1.5,0) -- (1.5,4.2) node[midway, right, xshift=0.5cm,line width=3pt] {$7\ast\frac{40}{3} km=\frac{280}{3} km$};
\end{tikzpicture}}
%Comparing Fractions
\subchapter{Comparing Fractions}
{Duckie knew that he could avoid the bounty hunters if he flew far enough in a single day. On the eighth day, Duckie and G{\"u}s traveled $\frac{1}{5}$ of the journey. If Duckie traveled more on the eighth day than the previous days ($\frac{1}{3}$ of the journey), the bounty hunter would stop chasing them. Did Duckie manage to confuse the bounty hunter?}
{Duckie realized that \begin{center}
    $\frac{5}{5}=1 \linebreak
    \frac{1}{3}\ast 1 = \frac{1}{3} \linebreak
    \frac{1}{3}\ast\frac{5}{5} = \frac{1\ast 5}{3\ast 5} = \frac{5}{15}\linebreak  $Which means $\frac{1}{3}$  is same as  $\frac{5}{15} \linebreak\linebreak$
    \textit{similarly}
   $ \frac{1}{5} = \frac{1}{5}\ast\frac{3}{3} = \frac{3}{15}$
\end{center}
\paragraph{} Now, we can compare the equal size parts together. $\frac{3}{15} < \frac{5}{15}$. Duckie traveled less on the eighth day than the other days combined.}
{We can only deal with fractions together using same-size parts. We can chop up one fraction by multiplying it by another fraction equal to one. If the \textit{denominator}s (the bottom number of the fractions) are the same, the fractions have same sized parts.}
{\begin{tikzpicture}
    \foreach \pos in {1,...,5} {	
        \draw[->,ultra thick,red, decorate, decoration={random steps,segment length=3pt,amplitude=0.2pt}] (0,\pos *3 - 3) -- (0,\pos *3) node[midway, left] {$\frac{1}{5} km$};
    }
    \foreach \pos in {1,...,3} {	
        \MODULO{\pos}{2}{\rval}
        \SUBTRACT{\pos}{1}{\bl}
        \MODULO{\bl}{2}{\bval}
        \definecolor{col}{rgb}{\rval,0,\bval}
        \draw[->,ultra thick,blue, decorate, decoration={random steps,segment length=3pt,amplitude=0.2pt}] (1,\pos *5 - 5) -- (1,\pos *5) node[midway, left] {$\frac{1}{3} km$};
    }

    \foreach \pos in {1,...,3} {	
        \MODULO{\pos}{2}{\rval}
        \SUBTRACT{\pos}{1}{\bl}
        \MODULO{\bl}{2}{\bval}
        \definecolor{col}{rgb}{\rval,0.7,\bval}
        \draw[->,ultra thick,purple, decorate, decoration={random steps,segment length=3pt,amplitude=0.2pt}] (3,\pos - 1) -- (3,\pos) node[midway, left] {$\frac{1}{15} km$};
    }
    \foreach \pos in {1,...,5} {	
        \draw[->,ultra thick,violet, decorate, decoration={random steps,segment length=3pt,amplitude=0.2pt}] (4,\pos - 1) -- (4,\pos) node[midway, left] {$\frac{1}{15} km$};
    }


\end{tikzpicture}
}
%Adding Fractions
\subchapter{Adding Fractions}
{Despite Duckie's best efforts, he was unfortunately not able to confuse the bounty hunter. He was starting to get worried, but G{\"u}s had a great idea. If they could find out where they were along the journey, they might have been able find a new route, which would take them away from the hunter's grasp! They traveled $\frac{1}{3}$, then they traveled $\frac{1}{5}$. How much of the journey have Duckie and G{\"u}s traveled?}
{Duckie and G{\"u}s have traveled $\frac{1}{3}+\frac{1}{5}\linebreak=\frac{3+5}{15}\linebreak=\frac{8}{15}$ of the journey.}
{When adding fractions, break up each fraction so that they are in terms of same-size parts (or having a common denominator). Then, add the numerators as usual.}
{\begin{tikzpicture}
    \coordinate (A) at (0,0);
    \coordinate (B) at (0,3);
    \coordinate (C) at (0,8);

    \node [inner sep=0pt] at (C) {\includegraphics[height=0.8cm]{DuckieGami}};
    \draw[-|,ultra thick,red, decorate, decoration={random steps,segment length=3pt,amplitude=0.2pt}] (A) -- (0,1) node[midway, left] {$\frac{1}{15} km$};
    \draw[-|,ultra thick,red, decorate, decoration={random steps,segment length=3pt,amplitude=0.2pt}] (0,1) -- (0,2) node[midway, left] {$\frac{1}{15} km$};
    \draw[->,ultra thick,red, decorate, decoration={random steps,segment length=3pt,amplitude=0.2pt}] (0,2) -- (B) node[midway, left] {$\frac{1}{15} km$};
    \draw[-|,ultra thick,blue, decorate, decoration={random steps,segment length=3pt,amplitude=0.2pt}] (B) -- (0,4) node[midway, left] {$\frac{1}{15} km$};
    \draw[-|,ultra thick,blue, decorate, decoration={random steps,segment length=3pt,amplitude=0.2pt}] (0,4) -- (0,5) node[midway, left] {$\frac{1}{15} km$};
    \draw[-|,ultra thick,blue, decorate, decoration={random steps,segment length=3pt,amplitude=0.2pt}](0,5) -- (0,6) node[midway, left] {$\frac{1}{15} km$};
    \draw[-|,ultra thick,blue, decorate, decoration={random steps,segment length=3pt,amplitude=0.2pt}] (0,6) -- (0,7) node[midway, left] {$\frac{1}{15} km$};
    \draw[->,ultra thick,blue, decorate, decoration={random steps,segment length=3pt,amplitude=0.2pt}] (0,7) -- (C) node[midway, left] {$\frac{1}{15} km$};

    \draw [decorate,
	decoration = {calligraphic brace,mirror,
		raise=10pt,amplitude=5pt}] (A) -- (B) node[midway, right, xshift=0.5cm,line width=3pt] {$\frac{3}{15} km$};
        \draw [decorate,
	decoration = {calligraphic brace,mirror,
		raise=10pt,amplitude=5pt}] (B) -- (C) node[midway, right, xshift=0.5cm,line width=3pt] {$\frac{5}{15} km$};
\end{tikzpicture}
}
\paragraph{} Duckie and G{\"u}s sighed in relief. After running for bounty hunters for two days and two nights, they were exhausted. They decided to travel slower for a few days and enjoy the scenery. 
\paragraph{} "I can't believe how persistent the goose authorities are." Duckie sighed. "I knew this journey would be tough, but I wasn't expecting bounty hunters!"
\paragraph{} G{\"u}s considered this. He had worked with the goose authorities, and knew their patterns. He realized how unusual this was. "I agree, the authorities must really be desparate".
\paragraph{} Just as Duckie opened his beak to reply, he felt a sudden weight on his shoulders, and began spiraling out of control. It was a net! Another bounty hunter must have set a trap! He started falling down fast, gaining speed as he went. \textbf{Crash}!. Duckie fell to the ground with a thud. G{\"u}s hurriedly landed next to him. 
\paragraph{} "Are you alright Duckie‽" he cried in alarm.
\paragraph{} "I'm all right, but I think I need to rest my wing for a few days", Duckie winced. He felt that he hadn't broken a wing, but saw some bruising, and knew he would need time to heal. 
 \paragraph{} "The forest is dangerous though! We can't stay here in the open with all the predators!"
 \paragraph{} "Last year, during migration, I made a friend, named Bessie the cow. I think her farm is close enough to walk to from here," Duckie said. "I think she will be able to help keep us safe from the goose authorities."
 \paragraph{} Together, they waddled to Bessie's farm.