\chapter{Sets and Functions}
\paragraph{} It was time for Duckie and Güs to continue on their journey. They had enjoyed spending time with Bessie, but they still needed to get to the Golden Goose.
\paragraph{} "What are we going to do about the Bounty Hunters?" Güs asked Duckie nervously. They both were concerned about the bounty hunters' traps.
\paragraph{} "I'm not sure" Duckie replied, "But we will have to keep moving. Hopefully we find something or someone along the way who can help".
\paragraph{} As the duo began to fly once more that evening, they saw an approaching flock of geese behind them.
\paragraph{} "Quick! Land!" Duckie cried. In a flash, in a rustle of feathers, the pair landed and waited, hoping to go unnoticed.
\paragraph{} There was the squaking of the the voices of the other group of geese as they landed beside Duckie and Güs.
\paragraph{} "Hi, how's it going?" one of the geese asked. This surprised Duckie. He was expecting to be taken away to the goose village, but he wasn't expecting a greeting.
\paragraph{} "Hello..." Duckie said cautiously. "Who are you? Have the goose authorities sent you?"
\paragraph{} "We are just a gaggle from the north. My name is Snow. What are the goose authoriies?". Duckie relaxed. Clearly, these geese wouldn't take them away. Güs, having had experience from journeying with the Goose Police, knew the value of traveling with a group.
\paragraph{} "Oh they aren't anyone important," Güs said to Snow. "We are trying to find the golden goose, and it seems like you are heading in the same direction as us. Mind if we join you?"
\paragraph{} "Alright!" Snow said. "But you'll have to join our dance!".
\vfill
\pagebreak
\subchapter{Cartesian Product}
{That night, the new flock decided to have their traditional nightly dance. It was a partners dance, with the following rules: geese with red feet were one group, and the geese with blue feet were another group. Each of the red geese partnered with each of the blue geese, where a red goose lead. The group of red geese in the dance was \{Güs, Snow, Grey, Blue\}. The group of blue geese was \{Duckie, Barnie, Swan\}. How many ways could the geese have formed pairs? What are those pairs?}
{All the pairs that the Geese could have formed were \{(Güs, Duckie), (Güs, Barnie), (Güs, Swan), (Snow, Duckie), (Snow, Barnie), (Snow, Swan), (Grey, Duckie), (Grey, Barnie), (Grey, Swan), (Blue, Duckie), (Blue, Barnie), (Blue, Swan)\}. There were 4 geese with red feet, and 3 geese with blue feet. There were \opmul{4}{3} ways to form pairs}
{A set is an unordered collection of things. If there are two sets, $A$ and $B$, the cartesian product ($AxB$) are the pairs $(a, b)$ where $a\in$ \footnote{$\in$, means "is in the set"} $A$ and $b\in B$.}
{}
\subchapter{Relations}
{Maggie, one of the blue footed geese and, the organizer of the dance, looked over Güs's shoulder to see what he was doing. \paragraph{} "Hi there! Whatcha doin'?" she asked inquisitively.
\paragraph{} "Nothing much, I'm just trying to figure out how the dance works. Can you take a look?"
\paragraph{} "Of course!" Maggie considered his drawings. "It seems like you are on the right track, but I think Snow and Barnie like to dance different styles. Maybe they shouldn't be a possible pair?"
\paragraph{} Güs removed their pairing from his plans. "So all of the pairs in the new set are in the original set."
\paragraph{} How many sets of pairs could Maggie and Güs make out of the original cartesian product?}
{Maggie and Güs can quite a few sets. These include \{\}, the $AxB$ itself, all sets with only a single pair, all the sets with only two pairs, etc.}
{If a set $A$ (also called the domain) is a subset of set $B$ (also called the codomain), then every element in set $A$ is also in set $B$. For every set $A$, there are $2^{|A|}$ possible subsets which can be made $\footnote{|A| means size of A}$. A subset of a cartesian product called is a relation.}
{}
\subchapter{Functions}
{Güs and Maggie continued to plan, and decided to come up with the list of pairs of geese. 
\paragraph{} "Hmm..." Maggie thought, looking at the list of relations. "There are not enough blue geese for each goose to have their own partner. How can we make sure each goose has their own partner?"
\paragraph{} Güs thought for a moment. "Maybe some of the blue geese can take turns switching between partners?"
\paragraph{} "I'm not sure, but maybe it would work. Let's try it out!"
Come up with a possible set of pairs dance partners Maggie and Güs could have formed.}
{Maggie and Güs eventually decided on the following set of dance partners \{(Güs, Barnie), (Snow, Barnie), (Grey, Swan), (Blue, Duckie)\} with Maggie sitting aside}
{A function is a special type of relation between the sets $A$ and $B$. In a function, in every pair $(a, b)$ (where $a\in A$ and $b\in B$), $a$ only maps to a single output. This allows us to say that the $f(a) = b$, or that the value $a$ maps to the value $b$.}
{}
\subchapter{Types of Functions}
{As the dance began, Maggie watched the other geese dance. Güs noticed her watching, and came over. 
\paragraph{} "Why don't you come over and dance with us?" Güs asked. 
\paragraph{} "Ah, you don't want me to dance, I don't to mess up the fun. Besides, this way, the blue geese don't have to worry about switching partners."
\paragraph{} Güs looked over. "Don't worry about that, it's fun! Besides, now there aren't enough red geese. C'mon!".
\paragraph{} Maggie hesitated, "Well, let's see how they do on their own for now."
\paragraph{} "You'll do great!"
\paragraph{} "Alright! Let's dance!" 
\paragraph{} Güs and Maggie went back to the dance floor together. 
What does the new function mapping red geese with blue geese look like?}
{With all the geese dancing, the red geese and new geese were mapped so that the dancing partners were \{(Güs, Maggie), (Snow, Barnie), (Grey, Swan), (Blue, Duckie)\}}
{A function where every element in the domain has a different mapping is called "one-to-one". A function where every element in the codomain is mapped to is called "onto". A function which is both one-to-one and onto is called "bijective".}
{}
\subchapter{Inverse of Functions}
{As the first dance came to a close, Maggie went to the makeshift stage to begin the next dance.  
\paragraph{} "All right everyone!" she announced happily. "Same dance, but this time, blue geese lead!" 
\paragraph{} The geese cheered, and began the dance again with the blue geese taking the lead. Was this new pairing a function? If so, what did it look like?}
{The new pairing was \{(Maggie, Güs), (Barnie, Snow), (Swan, Grey), (Duckie, Blue)\}. This new pairing was also a function, because every first element only mapped to a single output. For example, originally, $f($Güs$)= $Maggie, but now $f^{-1}($Maggie$) = $Güs}
{The inverse of a function $f(x)$ is written as $f^{-1}(x)$. The input of of $f$ is the output of $f^{-1}$, and the output of $f$ is $f^{-1}$. In other words, if $f(a)=b$, then $f^{-1}(b) = a$. In order for there to be an inverse function for $f(x)$, $f(x)$ must be bijective. }
{}
\subchapter{Continuous Functions}
{As Güs and Maggie danced with each other, they realized that they could also explain the movement of the dance as a function. $f(t) = -t\ast t+10$, where $t$ was the number of minutes since the dance began, and $f(t)$ represented how far they stood from the stage. What was Güs and Maggie's position when they had danced for 3 minutes?}
{$f(3) = -3\ast 3 + 10$\linebreak $f(3)=1$.\linebreak When they had danced for three minutes, Güs and Maggie were standing 1 meter from the stage.}
{Functions can be written as pair explicitly, or with a formula. When making a function using a formula, the pairs are $(x, f(x))$}
{}
